\begin{latexonly}
\newpage
\end{latexonly}
\subsection{sddssampledist}
\label{sddssampledist}

\begin{itemize}
\item {\bf description:}
{\tt sddssampledist} provides for psuedo-random sampling of probability
distributions.  It also provides nonrandom sampling, using Halton sequences.

\item {\bf example:} 
Draw some random samples from a normal (gaussian) distribution, G(z), shifted to have
a sigma of 10 and centroid of 5.
\begin{flushleft}{\tt
sddssampledist gaussian.sdds samples.sdds -samples=100 -columns=indep=z,df=G,output=zSample,factor=10,offset=5
}\end{flushleft}

\item {\bf synopsis:} 
\begin{flushleft}{\tt
sddssampledist [{\em input}] [{\em output}] [-pipe=[in][,out]]
-columns=independentVariable={\em name},{cdf={\em CDFName} | df={\em DFName}}
   [,output={\em name}][,units={\em string}][,factor={\em value}]
   [,offset={\em value}][,datafile={\em filename}]
   [,haltonRadix={\em primeNumber}[,randomize[,group={\em groupID}]]]
[-columns=...] [-samples={\em integer}] [-seed={\em integer}]
}\end{flushleft}

\item {\bf files:}

{\em input} is the default input file for distribution functions (DFs)
and cumulative distribution functions (CDFs).  {\em input} need not be
given if all {\tt -column} options give the {\tt datafile} qualifier.

{\em output} contains the samples.  The names of the sampled data are
by default the same as the names of the independent variable from the
{\tt -column} options.  These names may be changed by the {\tt output}
qualifier of that option.

\item {\bf switches:}
    \begin{itemize}
    \item \verb|pipe[=input][,output]| --- The standard SDDS Toolkit pipe option.
    \item {\tt columns=independentVariable={\em name},\{cdf={\em CDFName} | df={\em DFName}\}
[,output={\em name}][,units={\em string}][,factor={\em value}][,offset={\em value}]
[,datafile={\em filename}][,haltonRadix={\em primeNumber}[,randomize[,group={\em groupID}]]]} ---
   Any number this option may be given.
   Specifies the CDF or DF from which to draw samples ({\tt cdf} or {\tt df} qualifier), 
   as well as the variable that the CDF or
   DF depends on ({\tt independentVariable} qualifier).  The samples are values of this variable.
   {\tt output} allows specifying the name of the column for the samples, while {\tt units}
   allows specifying the units.  {\tt factor} and {\tt offset} may be used to perform a simple
   transformation of the sample values, according to $x \rightarrow x*f+o $.
   {\tt datafile} allows specifying an alternate file as the source for the distribution function
   data.  By default, the data is drawn from the main input file.
   {\tt haltonRadix} allows specifying the radix for generation of a non-random Halton sequence,
   which provides a much smoother sampling of the distribution than does a pseudo-random sequence.
   The radix should be a small prime number.  If you generate multiple sequences from the
   same radix, they will be correlated.  Hence, the {\tt randomize} qualifier should be used
   to remove the correlations.  If there are multiple {\tt column} options that should be
   randomized together (i.e., randomized relative to other data but not each other), the 
   {\tt group} qualifier can be used to assign these options to a specific *integer) group ID.
   \item \verb|samples| --- Specifies the number of samples to generate.
   \item \verb|seed| --- Specifies the seed for the random number generation.  Should be a large,
        odd integer.  If not given, the system clock is used to generate a seed.
    \end{itemize}
\item {\bf author:} M. Borland, ANL/APS.
\end{itemize}

