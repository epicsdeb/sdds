\begin{latexonly} 
\newpage 
\end{latexonly} 
\subsection{sdds2stream} 
\label{sdds2stream} 
 
\begin{itemize} 
\item {\bf description:} \hspace*{1mm}\\ 
{\tt sdds2stream} provides stream output to the standard output of data values from a group of columns or parameters.  
Each line of the output contains a different row of the tabular data or a different parameter.   
Values from different columns are separated by the delimiter string. 
If -page is not employed, all data pages are output sequentially.  
If multiple filenames are given, the files are processed sequentially in the order given. 
\item {\bf examples:} 
To output values of tunes for each page, one line per page: 
\begin{flushleft}{\tt 
sdds2stream APS.twi -parameters=nux,nuy -delimiter=" " 
}\end{flushleft} 
To output values of columns {\tt ElementName} and {\tt betax} for the first data page: 
\begin{flushleft}{\tt 
sdds2stream APS.twi -column=ElementName,betax -page=1 
}\end{flushleft} 
\item {\bf synopsis:}  
\begin{flushleft}{\tt 
sdds2stream \{{\em inputFileList} | -pipe[=input]\} 
[-page={\em pageNumber}] [-delimiter={\em delimitingString}]  
\{ -columns={\em columnName}[,{\em columnName}...] 
| -parameters={\em parameterName}[,{\em parameterName}...] 
| -arrays={\em arrayName}[,{\em arrayName}...] \} 
[-filenames] [-rows] [-noquotes] [-ignoreFormats] [-description]
}\end{flushleft} 
\item {\bf files:} 
{\em inputFileList} is a space-separated list of SDDS filenames.   
\item {\bf switches:} 
    \begin{itemize} 
    \item {\tt -pipe[=input]} --- The standard SDDS Toolkit pipe option. 
    \item {\tt -page={\em page-number} } --- Specifies the number of the data page for which output is desired.   
        Recall that pages are numbered sequentially beginning with 1.  More complete control of which pages 
        are output may be obtained using {\tt sddsconvert} or {\tt sddsprocess} as a filter. 
    \item {\tt -delimiter={\em delimitingString}} --- Specifies the delimiting string to be printed to  
        separate row entries or parameters. The 
        delimiter is printed with \verb|printf|, so that any of the usual escape sequences may be employed. 
    \item {\tt columns={\em columnName}[,{\em columnName}...]} ---  
        Specifies the names of the columns for which output is desired.  For each row of each 
        data page, the specified columns are printed on a single line, separated by the delimiting string.  The 
        default delimiting string is a single space. 
    \item {\tt -parameters={\em parameterName}[,{\em parameterName}...] } 
        --- Specifies the names of the parameters for which output is desired.  For each row of each 
        data page, the specified parameters are printed on a single line, separated by the delimiting string.  However, 
        since the default delimiting string is a newline, the parameters end up on separate lines. 
    \item {\tt arrays={\em arrayName}[,{\em arrayName}...]} ---  
        Specifies the names of the arrays for which output is desired.  
    \item {\tt filenames} --- Specifies that the filename will be printed out as each file is processed. 
    \item \verb|rows| --- Specifies that the number of rows per page for the tabular data section will 
        be printed out. 
    \item \verb|noquotes| --- Specifies that whitespace-containing string data will be printed without 
        the default double-quotes.  
    \item \verb|ignoreFormats| --- Specifies that the format data supplied in the file is to be
        ignored.  Guarantees printing of floating point data to full precision.
    \item \verb|description| --- Specifies printing of the description data for the data set.
    \end{itemize} 
\item {\bf see also:} 
        \begin{itemize} 
    \item \hyperref{Data for Examples}{Data for Examples (see }{)}{exampleData} 
        \item \progref{sddsprintout} 
        \item \progref{sddsconvert} 
        \item \progref{sddsprocess} 
        \end{itemize} 
\item {\bf author:} M. Borland, ANL/APS. 
\end{itemize} 
 
