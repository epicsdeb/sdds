\begin{latexonly}
\newpage
\end{latexonly}
\subsection{sddseventhist}
\label{sddseventhist}

\begin{itemize}
\item {\bf description:} 
{\tt sddseventhist} analyzes labeled events in a dataset to provide
histograms of the occurences of each type of event.  It can also
histogram the overlap off all types of events with a single type of
event.
\item {\bf synopsis:} 
\begin{flushleft}{\tt
sddseventhist [-pipe=[input][,output]] [{\em inputFile}] [{\em outputFile}]
-dataColumn={\em columnName} -eventIdentifier={\em columnName} [-overlapEvent={\em eventValue}]
[{-bins={\em number} | -sizeOfBins={\em value}}] 
[-lowerLimit={\em value}] [-upperLimit={\em value}] 
[-sides] [-normalize[=\{sum | area | peak\}]] 
}\end{flushleft}
\item {\bf files:}
{\em inputFile} is a file containing at least two columns of data.
One column must contain string entries that serve as ``event
identifiers''; for example, these might be the names of channels that
issued an alarm.  The other column must contain numerical data that
will be histogrammed; for example, these might be the times at which
alarms occured.  The {\em outputFile} contains one histogram of this
numerical data for each unique value in of the event identifier; the
histogram contains only the data that matches that identifier.
\item {\bf switches:}
    \begin{itemize}
    \item \verb|-pipe[=input][,output]| --- The standard SDDS Toolkit pipe option.
    \item {\tt -dataColumn={\em columnName}} --- Specifies the name of the data column to be histogrammed.
    \item {\tt -eventIdentifier={\em columnName}} --- Specifies the name of the string column that
        identifies events.
    \item {\tt -overlapEvent={\em eventValue}} --- Requests computation of the overlap of the
        histograms of each event with the histogram of event {\em eventValue}.  Useful in determining
        which events always occur at the same time as event {\em eventValue}.
    \item {\tt -bins={\em number}} --- Specifies the number of bins to use.  The default is 20.
    \item {\tt -sizeOfBins={\em value}} --- Specifies the size of bins to use.  The number of bins is
        computed from the range of the data.
    \item {\tt -lowerLimit={\em value}} --- Specifies the lower limit of the histogram.  By default,
        the lower limit is the minimum value in the data.
    \item {\tt -upperLimit={\em value}} --- Specifies the upper limit of the histogram.  By default,
        the upper limit is the maximum value in the data.
    \item {\tt -sides} --- Specifies that zero-height bins should be attached to the lower
        and upper ends of the eventhistogram.  Many prefer the way this looks on a graph.
    \item {\tt -normalize[=\{sum | area | peak\}]} --- Specifies that the histogram should be normalized, and how.
        The default is {\tt sum}.  {\tt sum} normalization means that the sum of the heights will be 1.
        {\tt area} normalization means that the area under the histogram will be 1.
        {\tt peak} normalization means that the maximum height will be 1.
    \end{itemize}
\item {\bf see also:}
    \begin{itemize}
    \item \progref{sddscorrelate}
    \item \progref{sddshist}
    \item \progref{sddshist2d}
    \end{itemize}
\item {\bf author:} M. Borland, ANL/APS.
\end{itemize}

