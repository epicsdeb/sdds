\begin{latexonly}
\newpage
\end{latexonly}
\subsection{sddsnormalize}
\label{sddsnormalize}

\begin{itemize}
\item {\bf description:} \verb|sddsnormalize| performs various normalizations of
column data.
\item {\bf synopsis:} 
\begin{flushleft}{\tt
sddsnormalize [{\em inputFile}] [{\em outputFile}] [-pipe[=input][,output]] \\ \ 
-columns=[mode={\em mode}][,suffix={\em string}][,exclude={\em wildcardString}],{columnName}[,{columnName}...]
}\end{flushleft}
where mode is one of {\tt minimum}, {\tt maximum}, {\tt largest}, {\tt signedLargest},
or {\tt spread}, referring to the factor used for normalization (see below).
\item {\bf files:}
{\em inputFile} is an SDDS file containing data to be processed.  The {\em outputFile} argument is
optional.  If it is not given, and if an output pipe is not selected, then the input file will be
replaced.
\item {\bf switches:}
    \begin{itemize}
    \item {\tt -pipe=[input][,output] } --- The standard SDDS Toolkit pipe option.
    \item {\tt -columns=[mode={\em mode}][,suffix={\em string}]} 
        {\tt [,exclude={\em wildcardString}],{columnName}[,{columnName}...]} --- 
Any number of these options may be given.
Each specifies columns to normalize and in what {\em mode}.  {\em mode} may be
one of {\tt minimum}, {\tt maximum}, {\tt largest}, {\tt signedLargest},
or {\tt spread}, referring to the factor used for normalization:
largest (the default) is the maximum absolute value; signed largest is the
same value, but with the sign restored; spread is the maximum minus the minimum.
Each {\em columnName} qualifier gives a possibly wildcarded string specifying
columns to normalize.  {\tt exclude} may be used to exclude columns from normalization
that are matched by a {\em columnName}.
The {\tt suffix} qualifier optionally specifies a suffix to be appended to each
column name, to create a new column for the output file; if not given, then the
original data are replaced with the normalized data.
    \end{itemize}
\item {\bf author:} M. Borland, ANL/APS.
\end{itemize}


